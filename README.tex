\documentclass[a4paper]{article}
\usepackage[T1]{fontenc}
\usepackage[a5paper,nohead,foot=1cm]{geometry}
\pagestyle{plain}
\usepackage[utf8]{inputenc}
\fussy
\usepackage[russian]{babel}
\frenchspacing
\title{Информационный проект \flqq{}оптимизация фильтрации текстовых данных с \
	помощью низкоуровневого программировния в среде GNU/Linux\frqq{}}
\author{Гордиенков Захар, 10 класс}
\date{\today}
\begin{document}

\maketitle
\pagebreak
{\small\tableofcontents}
\pagebreak

Цель: создать средство фильтрации текстовых данных (в т.ч. системных логов),
которое будет превосходить уже существующие решения в
производительности

Задачи:
\begin{itemize}
	\item Изучить базовые концепции низкоуровневого программирования
	\item Изучить синтаксис и особенности языка ассемблера
	\item Разобраться, как работает поиск по тексту, в том числе регулярные
	выражения
	\item Рассмотреть доступные средства фильтрации текстовых данных в
	системах GNU/Linux
	\item Создать свою более производительную систему фильтров
\end{itemize}

\pagebreak

\section{Глоссарий}
	\begin{itemize}
		\item системы GNU/Linux (также просто Linux) -- семейство
			свободных Unix-подобных операционных систем, основанное
			на утилитах операционной системы GNU, разработанной
			Ричардом Мэтью Столлманом и ядре Linux, создателем
			которой является Линус Торвальдс
		\item Ассемблер -- (не путать с языком ассемблера, который часто
			называют просто ассемблером) - транслятор программы из
			текста на языке ассемблера в программу на машинном
			языке (объектный файл). Самые известные ассемблеры на
			данный момент -- TASM, FASM, NASM и MASM. Не путать с
			языком ассемблера, который зачастую называют ассемблером
		\item Ассемблирование -- процесс работы ассемблера
		\item Язык ассемблера (англ. assmebly language) -- представление
			команд процессора в понятном для человека виде.
			Большинство ассемблеров имеют свои синтаксические
			особенности при записи процессорных команд. Не путать с
			ассемблером
		\item Компоновщик (редактор связей, англ. linker) -- прикладная
			программа, которая собирает исполняемый файл из одного
			или нескольких объектных модулей
		\item Исполняемый файл -- машинный код, который предназначен
			для чтения и выполнения процессором. В Unix-системах
			для таких файлов используется формат ELF (см. ниже), в
			Windows - PE (Portable Executable)
		\item Регулярные выражения (англ. regular expressions) --
			формальный язык, используемый в программах, работающих с
			текстовыми данными, для поиска подстрок и манипуляций с
			ними, в основе которого - использование символов
			подстановки (символ-джокер, англ. wildcard character).
			Строка, по которой осуществляется поиск, называется
			маской (шаблон, англ. pattern)
		\item Объектный файл -- файл, содержащий представление части
			программы. Предназначен для соединения с другими
			объектными файлами в исполнимый модуль (или библиотеку)
			с помощью компоновщика. Объектные файлы в Linux имеют
			формат ELF, в Windows - COFF
		\item ELF, Executable \& Linlable Format -- формат исполнимых и
			компонуемых файлов, используемый во многих
			Unix-подобных системах. ELF-64 -- формат ELF,
			адаптируемый под 64-разрядную архитектуру
		\item Unix, Unix-подобные системы -- группа
			многопользовательских многозадачных переносимых
			операционных систем, в основе которых лежит одноимённый
			проект компании AT\&T.
		\item Intel-синтаксис -- формат записи инструкций процессора,
			используемый в документации к продуктам компании Intel.
		\item Разрядность процессора -- размер (в битах) машинного
			слова в таком процессоре
	\end{itemize}
\pagebreak
\section{Вступление}
\section{Продукт}
\section{Вывод}
\section{Список литературы}
	\begin{itemize}
		\item Э. Немет, Г. Снайдер, Т. Хейн, Б. Уэйли, Д. Макин --
			\flqq{}Unix и Linux: руководство системного
			администратора, 5-е издание \frqq{}; СПб: ООО
			`Диалектика`, 2018. 1168 страниц
		\item Зубков С. В. -- \flqq{}Assembler. Для DOS, Windows и
			Unix \frqq{}; М: ДМК Пресс, 2017. 638 страниц
		\item Robert C. Martin -- \flqq{}Clean Architecture. A Craftsman's
			Guide to Software Structure and Design \frqq{}; Pearson
			Education Inc, 2018. 350 страниц
		\item Документ \flqq{}Intel® 64 and IA-32 Architectures:
			Software Developer’s Manual \frqq{}; Май 2007
		\item Документ \flqq{} INTEL 80386. PROGRAMMER'S REFERENCE
			MANUAL\frqq{}; 1986. 421 страница
	\end{itemize}
	Искренне благодарен авторам вышеперечисленных книг, статей и
	видеоматериалов.
\end{document}
