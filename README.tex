\documentclass[a4paper]{article}
\usepackage[T1]{fontenc}
\usepackage[bindingoffset=0cm,bottom=2cm,a5paper,nohead,foot=1cm]{geometry}
%\usepackage[left=2cm,right=2cm,
%    top=2cm,bottom=2cm,bindingoffset=0cm]{geometry}
\pagestyle{plain}
\usepackage[utf8]{inputenc}
\fussy
\usepackage[russian]{babel}
\frenchspacing
\title{Информационный проект \flqq{}оптимизация фильтрации текстовых данных с \
	помощью низкоуровневого программировния в среде GNU/Linux\frqq{}}
\author{%Работу выполнил:\\
Гордиенков Захар Юрьевич}
%Ученик 10 \flqq{}А\frqq{} класса\\
%ГБОУ СОШ №257\\
%Руководитель: Мирошниченко Оксана Сергеевна\\
%Санкт-Петербург\\
%2023}
\date{\today}
\begin{document}

\maketitle
\pagebreak
{\tableofcontents}
\pagebreak

Цель: создать средство фильтрации текстовых данных (в т.ч. системных логов),
которое будет превосходить уже существующие решения в
производительности

Задачи:
\begin{itemize}
	\item Изучить базовые концепции низкоуровневого программирования
	\item Изучить синтаксис и особенности языка ассемблера
	\item Рассмотреть доступные средства фильтрации текстовых данных в
	системах GNU/Linux
	\item Создать свою более производительную систему фильтров
\end{itemize}

\pagebreak
\section{Вступление}
	Малая эффективность средств текстовой фильтрации является актуальной
	проблемой в сфере системного администрирования. В частности, сложность
	использования, отсутствие гибкости и низкая скорость работы. Считается,
	что при использовании языков низкого уровня можно оптимизировать как
	использование системных ресурсов, так и сократить время выполнения
	программы за счёт прямого доступа к аппаратным компонентам системы и
	возможности оптимизации под определённую системную архитектуру.
	Следовательно, можно предположить, что с помощью низкоуровнего
	программирования можно решить проблему низкой скорости работы при
	фильтрации текстовых данных. Такое ПО могло бы быть востребованным.

	В рамках этого проекта изучался ассемблер NASM
	\footnote{NASM (Netwide Assembler) 2.16.01, автором которого является
	британский программист Саймон Тэтхем (Simon Tatham), распространяется
	по упрощённой лицензии BSD (Simplified (2-clause) BSD License)},
	который работает на аппаратной архитектуре x86 16-и, 32-х и 64-битной
	разрядности. Ассемблер до сих пор поддерживается разработчиками, имеет
	избыточную документацию, является кроссплатформенным, поэтому при выборе
	инструментов для создания продукта было выбрано именно это ПО. В рамках
	этого проекта для простоты рассматривался только набор инструкций,
	поддерживаемый 32-битной архитектурой.
\pagebreak
\section{Концепция низкоуровнего программирования}
	\subsection{История}
	\subsection{Современность}
\section{Особенности языка Ассемблера}
	\subsection{Команды}
	\subsection{Процесс сборки исполняемого файла}
\section{Отчёт о работе над продуктом}
\pagebreak
\section{Вывод}
\pagebreak
\section{Список литературы}
	\begin{itemize}
		\item Э. Немет, Г. Снайдер, Т. Хейн, Б. Уэйли, Д. Макин --
			\flqq{}Unix и Linux: руководство системного
			администратора, 5-е издание \frqq{}; СПб: ООО
			`Диалектика`, 2018. 1168 страниц
		\item Зубков С. В. -- \flqq{}Assembler. Для DOS, Windows и
			Unix \frqq{}; М: ДМК Пресс, 2017. 638 страниц
		\item Robert C. Martin -- \flqq{}Clean Architecture. A Craftsman's
			Guide to Software Structure and Design \frqq{}; Pearson
			Education Inc, 2018. 350 страниц
		\item Документ \flqq{}Intel® 64 and IA-32 Architectures:
			Software Developer’s Manual \frqq{}; Май 2007
		\item Документ \flqq{} INTEL 80386. PROGRAMMER'S REFERENCE
		MANUAL\frqq{}; 1986. 421 страница
	\end{itemize}
	Искренне благодарен авторам вышеперечисленных книг, статей и
	видеоматериалов.


\pagebreak
\section{Приложения}
	\subsection{Глоссарий}
	\begin{itemize}
		\item системы GNU/Linux (также просто Linux) -- семейство
			свободных Unix-подобных операционных систем, основанное
			на утилитах операционной системы GNU, разработанной
			Ричардом Мэтью Столлманом и ядре Linux, создателем
			которой является Линус Торвальдс
		\item Ассемблер -- (не путать с языком ассемблера, который часто
			называют просто ассемблером) - транслятор программы из
			текста на языке ассемблера в программу на машинном
			языке (объектный файл). Самые известные ассемблеры на
			данный момент -- TASM, FASM, NASM и MASM. Не путать с
			языком ассемблера, который зачастую называют ассемблером
		\item Ассемблирование -- процесс работы ассемблера
		\item Язык ассемблера (англ. assmebly language) -- представление
			команд процессора в понятном для человека виде.
			Большинство ассемблеров имеют свои синтаксические
			особенности при записи процессорных команд. Не путать с
			ассемблером
		\item Компоновщик (редактор связей, англ. linker) -- прикладная
			программа, которая собирает исполняемый файл из одного
			или нескольких объектных модулей
		\item Исполняемый файл -- машинный код, который предназначен
			для чтения и выполнения процессором. В Unix-системах
			для таких файлов используется формат ELF (см. ниже), в
			Windows - PE (Portable Executable)
		\item Регулярные выражения (англ. regular expressions) --
			формальный язык, используемый в программах, работающих с
			текстовыми данными, для поиска подстрок и манипуляций с
			ними, в основе которого - использование символов
			подстановки (символ-джокер, англ. wildcard character).
			Строка, по которой осуществляется поиск, называется
			маской (шаблон, англ. pattern)
		\item Объектный файл -- файл, содержащий представление части
			программы. Предназначен для соединения с другими
			объектными файлами в исполнимый модуль (или библиотеку)
			с помощью компоновщика. Объектные файлы в Linux имеют
			формат ELF, в Windows - COFF
		\item ELF, Executable \& Linlable Format -- формат исполнимых и
			компонуемых файлов, используемый во многих
			Unix-подобных системах. ELF-64 -- формат ELF,
			адаптируемый под 64-разрядную архитектуру
		\item Unix, Unix-подобные системы -- группа
			многопользовательских многозадачных переносимых
			операционных систем, в основе которых лежит одноимённый
			проект компании AT\&T.
		\item Intel-синтаксис -- формат записи инструкций процессора,
			используемый в документации к продуктам компании Intel.
		\item Разрядность процессора -- размер (в битах) машинного
			слова в таком процессоре
		\item Лицензия BSD (Berkeley Software Distribution license) --
			программная лицензия университета Беркли. Этот документ
			впервые был применён при распространении UNIX-подобных
			операционных систем BSD
		\item x86 -- архитектура процессора и набор команд, созданный
			компанией Intel
		\item Архитектура процессора -- структурное устройство
			процессора. Процессоры одной архитектуры устроены схожим
			образом и имеют одинаковый набор машинных команд.
			Например, ARM (часто встречается среди мобильных
			устройств), RISC-V (экспериментальный проект, активно
			развивающийся в последние годы), x86-64 (характерна для
			современных персональных компьютеров).  \end{itemize}
\end{document}
